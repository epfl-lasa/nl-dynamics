% -*- mode:LaTeX; mode:visual-line; mode:flyspell; fill-column:75 -*-
\section{Conclusions and Future Work}
\label{secConclusions}


In this work, we have demonstrated how a robot manipulator can learn to modify its Dynamical System using natural language input from a user supervising the robot, enabling the robot to adapt to changing environments.
After the corrective model has been trained, this method offers an alternative teaching interface
that removes the need for physical interaction during task refinement, and provides an intuitive interface for controlling the robot suitable for lay users.

Future work will address using \emph{features} to further enable generalization to different scenarios.
For example, the commands ``keep the tool upright'' or ``stay above the table'' are respectively defined over features of the robot and world states, and these features should be used when generating the correction.

%Impedance control: change the stiffness of the robot during the task execution.
