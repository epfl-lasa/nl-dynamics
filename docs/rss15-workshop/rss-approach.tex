% -*- mode:LaTeX; mode:visual-line; mode:flyspell; fill-column:75 -*-

\section{Approach}
\label{sec:approach}

Our approach performs online modifications of a Dynamical System (DS) trajectory generator, applied to a robot manipulator performing some reaching task (bringing the end effector to a particular goal location).
Such tasks can be conveniently represented by a DS whose trajectories converge to the target point.
However, some tasks may require that not only the goal but also the path to the goal is important.
We leverage the fact that the user can show a demonstration of the desired behavior along with a natural language description of this correction.
We train a model to generalize from the demonstrations to other scenarios, such that the user can simply ``tell'' the robot how to change their behavior in novel domains without the need for any kinesthetic corrections.

We will now describe our trajectory generator formulation as a Locally Modulated Dynamical System (LMDS), how the user provides demonstrations of the desired (correct) behavior along with a natural language description, and how to infer a new controller solely from natural language instructions given during execution.

\subsection{Locally Modulated Dynamical Systems}

We model the desired motion of the robot with a first order autonomous DS:
\begin{equation}
  \label{eq:DS_general}
  \dot x = f(x)
\end{equation}
where $x \in \mathbb{R}^3$ is the position of the end-effector of the robot with respect to an external coordinate frame.
We will exclusively consider DS models that are stable at a single attractor point.
Specifically, we use Locally Modulated Dynamical Systems (LMDS), which uses a basic DS model with known stability properties $\dot x = f^o(x)$ and achieves flexibility by reshaping this model locally:
\begin{equation}
  \label{eq:DS_reshaped}
  \dot x = f(x) = M(x)f^o(x),
\end{equation}
where $M(x)$ is a matrix-valued function that rotates and scales the original dynamics in a continuous manner across the workspace.
An example is given in \Cref{figProblemSetup}, which plots the integral curves of a DS over 2d space.
In this case, the original dynamics is a linear system and the DS is locally rotated in the right part of workspace.
The modulation function $M$ is learned from incoming trajectory data by using Gaussian Process Regression (GPR).
 This procedure is described in detail in \cite{Kronander2015}.


\subsection{Demonstrations}

During training, the user can provide kinesthetic demonstrations (by back-driving the robot) of the correct behavior and attach a natural language utterance to the demonstration.
For example, the user can say ``come from above'' while back-driving the robot along an upwards trajectory.
More formally, each demonstration $\xi$ for an utterance $\Lambda$ consists of a sequence of $N$ demonstrated positions and velocities ($x$ and $\dot{x}$):
\begin{equation}
  \xi_\Lambda = \braces{x_i, \dot{x}_i}_{i=0}^N.
\end{equation}
In our experiments we train a model containing several demonstrations provided by the user.


\subsection{Inferring Controllers from Language}

During execution of the task, the robot can locally modify its trajectory generation model~($M\parens{x}$ in \cref{eq:DS_reshaped}) when it receives a natural language utterance from the user.
To do so, when the user provides an utterance (representing the desired correction), we apply a correction $\Delta_\Lambda$ from the current robot position, by translating the demonstrated poses:
\begin{equation}
  \Delta_\Lambda = \braces{x_i - x_0  + x_r, \,\, \dot{x}_i}_{i=0}^N,
\end{equation}
where $x_r$ is robot pose when the utterance was initiated,
and $x_i-x_0$ represents the transformation of the trajectory data to the new location (this enables our approach to generalize to new poses in the state).

This correction $\Delta_\Lambda$ is used as an input to the learning algorithm presented in \citet{Kronander2015} to yield the modulation matrix $M\parens{x}$ in \Cref{eq:DS_reshaped}.
This results in an immediate modification of the robot trajectory according to the utterance provided by the user.
These are remembered during subsequent task executions, and the user can provide further natural language modifications.
