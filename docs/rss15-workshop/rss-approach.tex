% -*- mode:LaTeX; mode:visual-line; mode:flyspell; fill-column:75 -*-


\section{Approach}
\label{sec:approach}

Our approach performs online modifications of a Dynamical System (DS) trajectory generator, applied to a robot manipulator performing a reaching task (bringing the end effector to a particular goal location).
In many tasks the original dynamics are not enough to successfully complete the task, for example if the user wants the robot to take a specific path.
We leverage the fact that the user can show a demonstration of the desired behavior along with a natural language description of this correction.
We train a model to generalize from the demonstrations to other scenarios, such that the user can simply ``tell'' the robot how to change their behavior in novel domains without the need for any kinesthetic corrections.

We will now describe our trajectory generator formulation as a Modified Dynamical System, how the user provides demonstrations of the desired (correct) behavior along with a natural language description, and how to infer a new controller solely from natural language instructions given during execution.

\subsection{Modified Dynamical System}

We model the desired motion of the robot with a first order autonomous DS:
\begin{equation}
  \label{eq:DS_general}
  \dot x = f(x)
\end{equation}
where $x \in \mathbb{R}^3$ is the position of the end-effector of the robot with respect to an external coordinate frame. We will exclusively consider DS models that are stable at a single attractor point. Specifically, we use Locally Modulated Dynamical Systems (LMDS), which uses a basic DS model with known stability properties $\dot x = f^o(x)$ and achieves flexibility by reshaping this model locally:
\begin{equation}
  \label{eq:DS_reshaped}
  \dot x = f(x) = M(x)f^o(x)
\end{equation}
where $M(x)$ is a matrix valued function that rotates and scales the original dynamics in a continuous manner across the workspace. An example is given in \Cref{figProblemSetup}, which plots the integral curves of a DS over 2d space. In this case, the original dynamics is a linear system and the DS is locally rotated in the right part of workspace. The procedure of reshaping the DS such that it locally aligns with incrementally-arriving training data is described in detail in \todo{add reference to klas's RAS paper}.


\subsection{Demonstrations}

During training, the user can provide kinesthetic demonstrations (by back-driving the robot) of the correct behavior and attach a natural language utterance to the demonstration.
For example, the user can say ``come from above'' while back-driving the robot along an upwards trajectory.


In our experiments we train a model containing several corrections the user may describe.


\subsection{Inferring Controllers from Language}

During execution of the task, the robot can locally modify its trajectory generation model~(\cref{eq:DS_reshaped}) when it receives a natural language utterance from the user.
