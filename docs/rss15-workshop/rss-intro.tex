% -*- mode:LaTeX; mode:visual-line; mode:flyspell; fill-column:75 -*-
\section{Introduction}
\label{secIntroduction}


Enabling robots to understand natural language would provide an intuitive interface between lay users and complex robots.
Prior work on enabling robots to understand language has addressed diverse tasks such as following route directions through indoor and outdoor environments, manipulating objects,


These approaches map language to robot plans or high-level symbolic control languages.
This is effective for tasks that require only high-level reasoning, but requires a low-level controller that can execute the inferred actions.

However, there are many robot tasks that require reasoning not just about \emph{what} to do, but \emph{how} to do it, for example fine manipulation tasks or tasks that require varying impedance during execution.
To date, no work connects language and robot behavior at the controller\todo{??} level.
Using a controller such as a Dynamical System provides several key benefits:
\begin{itemize}
\item defined over the entire state space: robust to perturbations of the robot.
\item reactive behavior: robust to perturbations of the environment.
\item generalizability: a dynamical system provides a family of trajectories, not just a single one.
\end{itemize}



In this work, we enable the robot to understand \emph{how} it should do a task using natural language instructions.

Specifically, we represent a task using a ``vanilla'' DS, and leverage the expressive power of language to infer \emph{modifications} to the Dynamical System.
We train a model from human demonstrations of the desired behavior modifications (using kinesthetic teaching), and generalize those to novel states in the world.
During execution, the user can then change the behavior of robot (through the controller) by using the trained natural language commands.



\begin{figure}[t]
  \centering
  \missingfigure{Modified DS}
  \caption{A Modified Dynamical System, representing the controller policy at every point in the state space (2D in this case): at every point the robot follows the gradient until it reaches the global minima.
 In the colored areas, the user has provided a correction to the policy (red line), which updates the policy in the same region of state space.
    Our goal is to use natural language to modify a dynamical system, resulting in a controller that performs the desired behavior.
}
  \label{figProblemSetup}
\end{figure}



For example, an assistive robot may need to understand how much pressure it should exert when cleaning a patient's skin,
or a factory robot may need to change how stiffly it should hold a part when cooperatively holding it for a person to assemble,
or a <> robot may need to know from which direction to approach an object
