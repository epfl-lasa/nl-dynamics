% -*- mode:LaTeX; mode:visual-line; mode:flyspell; fill-column:75 -*-
\section{Introduction}
\label{secIntroduction}



Enabling robots to understand natural language would provide an intuitive interface between lay users and complex robots.

Prior work on enabling robots to understand language has addressed diverse tasks such as following route directions through indoor and outdoor environments, manipulating objects,



requiring

This specifies \emph{what} the robot should do to complete a task.




In this work, we enable the robot to understand \emph{how} it should do a task using natural language instructions.


We train a model from human demonstrations of t


For example, an assistive robot may need to understand how much pressure it should exert when cleaning a patient's skin,
or a factory robot may need to change how stiffly it should hold a part when cooperatively holding it for a person to assemble,
or a <> robot may need to know from which direction to approach an object



Furthermore, using controllers provides several key benefits:
 - defined over the entire state space: robust to perturbations of the robot.
 - reactive behavior: robust to perturbations of the environment.

\begin{figure}[h]
  \centering
  \missingfigure{Modified DS}
  \caption{A Modified Dynamical System, representing the controller policy at every point in the state space (2D in this case). In the colored areas, the user has provided a correction to the policy (red line), which updates the policy in the same region of state space.
    Our goal is to use natural language to modify a dynamical system, resulting in a controller that performs the desired behavior.
}
  \label{figProblemSetup}
\end{figure}
