Finally this project has been developed but there is still some place for improvement. We took in consideration all the feedbacks we get from the two users who tried the program and targeted some problem we could have solved if the time was on our side. 

First of all, considering the accuracy of the Pocketsphinx algorithm and because some user couldn't get understand by it and so were blocked into one state by feeling a bit lost, we could simply let the robot repeat the state in which it is and/or the instructions every periodic amount of time (ex: 10 seconds). Moreover this function has already been developed (cf. Giving Command Machine), we should just add it in every states. More generally, some functions we developed could have been re-used a bit everywhere in the code. 

Another important thing we could have done better is to grow the user experiment (System Usability Scale) to a bigger number of subject to collect more feedback and so make a real conclusion out of it. In fact the product was designed to be use by everyone not prepared.

Concerning the Giving Command Machine, we could have simply add a functionality which could have allow the user to say command one after another without doing the full branch each time. In fact one of the feedback was also that it was kind of annoying to do the full branch for only one command. This function could have been easily developed.

As explained during the System Usability Experience, one of the user was stuck at one point because he couldn't get understand by the algorithm to teach a command. In order to face this problem we could simply use a more accurate speech-to-text algorithm which could allow us to put a lot more words without the fear of having a confusion between two similar words. It could unblocked the situation were a user is misunderstand by using the one or two keywords selected to pass a state.

