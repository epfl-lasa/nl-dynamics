\subsection{General Discussion}
Regarding the project in general we succeeded to develop a dialogue between the robot and a user. 
By looking at the final version of the program we can say that a dialogue has been successfully created between a robot and a user. Moreover if we look at the specifications we can also say that they have been implemented despite the fact that some can be improved.

In order to make the use of the robot intuitive and easy to use without preliminary explanation we need to use the speech feedback of the robot. In fact this functionality help a lot the understanding and the step the user has to follow to control well the robot.

The structure of the dialogue has been optimized, every time the user is saying something the robot will acknowledge it. This functionality is very important, so then the user can realize if the robot understood well or not the saying of the user. Sometimes the command said by the user was even said back by the robot with the help of the global variable.

Close to the end of the semester we did the Confusion Matrix Experiment. This experiment helped us finding the best keywords we could use in our program so then the algorithm has the best chance to understand well the word whichever the accent the user has. The choice of the keywords we selected was based on the success rate of the recognition of the word by the algorithm, but also the distinctiveness with the other words in order to avoid confusion. 

In order to make this program usable on every robot we decided to develop a main structure in which everything could be the same except the converter that will convert the type we use in the State Machine (String) into the type the robot we want to use is using. 

Finally we had the chance to realize an user experience and get feedback out of it. Some of the feedback that came out were that the algorithm didn't really succeed to understand what they were saying, that led to the problem that the user felt lost because the robot weren't acknowledging anything for the reason that users were stuck in the same state, we will propose a solution for the problem in the Improvements part. The general feedback was that it was intuitive and easy to use but still needed some help from the experimenter at the beginning. \\

\subsection{Kuka Implementation}
The final step of the project was the implementation on the Kuka and it was a success. We only had to develop the converter for the Kuka in order to make it work.
We didn't stop here, in fact my project and the one of S. Ballmer were linked. S. Ballmer was developing an algorithm enabling the robot to learn movements by providing good transformed data. Assembling the two programs was the longest part of the implementation on the Kuka.
We finally made it work and we could record some video of the Kuka working.